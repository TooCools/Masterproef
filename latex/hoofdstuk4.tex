\chapter{Discussie}
\section{Algoritmes}
Het vorige hoofdstuk toont de resultaten van de verschillende algoritmes. RF en DT presteren duidelijk beter dan beide types van PA. DT presteert goed, maar is nog steeds bekend om het slecht omgaan met ruis en het overfitten. Aangezien het verschil in uitvoeringstijd tussen DT en RF minimaal is, is het altijd beter om een RF te gebruiken. Alhoewel in de test blijkt dat de grootte van het RF geen al te grote impact heeft op de uitvoeringstijd, kan dit mogelijk nog een probleem vormen door de beperkte rekenkracht van de Raspberry Pi. Het controleprogramma voor de fiets zal immers parallel lopen met de cadanscontroller. Gelukkig heeft de Raspberry Pi vier cores en is een RF makkelijk te parallelliseren (door het instellen van het aantal parallelle taken).
\\\\
Voor een RF zijn volgende hyperparameters aangeraden:
\begin{gather*}
\text{RF \tab diepte=4 of 5, aantal bomen=10-20}
\end{gather*}
Een groot nadeel van RF is dat het niet goed om kan met ongeziene omstandigheden. Bijvoorbeeld als het algoritme van niets of weinig data begint, kan het niet uit de reeds geziene data een goede voorspelling maken (t.o.v. lineaire regressie voor een simpel lineair probleem).
\section{Waarheid}
In de verschillende testen werd het fietsersmodel altijd als waarheid (\textit{ground truth}) gezien. Dit kan echter niet gebruikt worden aangezien dit ongekend is. Als de fietser een update wilt doen, dan weet het algoritme niet hoe hard het algoritme moet aanpassen. Als de fietser sneller wilt trappen, is het dan twee rpm sneller of vijf? Dit is in deze thesis niet onderzocht, maar vormt nog een potentieel probleem in het succes van de real-time cadansaanpassing in een automatische fiets transmissie.
\\\\
Het voornaamste probleem dat zich hier voordoet, is dat twee achtereenvolgende trainingen data zullen genereren die in conflict gaan met elkaar. Stel dat de fietser 60 rpm trapt. De eerste keer dat hij op de knop duwt, wordt er trainingsdata gegenreerd met als doelvariabele 65 rpm. De tweede keer wordt er gelijkaardige data gegenereerd, maar dan met een doelvariabele van 70 rpm. Een mogelijke oplossing voor dit probleem, is het kijken naar de dichtst bijzijnde buren in de trainingsset en de doelvariabelen te updaten volgens de afstand. Hoeveel de doelvariabele bijgewerkt moet worden, zou exponentieel moeten dalen voor observaties die ver weg liggen.
\section{Conceptuele drift}
In sectie 3.4 werden de resultaten van twee algoritmes, statisch schuivend venster en bevoordeelde reservoir sampling, getoond. In beide gevallen moest er minstens drie keer zoveel geleerd worden om het nieuwe concept te leren, wat toch wel een groot verschil is met het beginnen vanaf nul. Deze technieken doen het alleszins wel beter dan geen “vergeet” algoritme. Het voornaamste probleem hier is dat mensen die deze fiets tweedehands kopen of als leen-, familiefiets gebruiken, een slechtere ervaring zullen hebben dan mensen die de fiets nieuw kopen en enkel zelf gebruiken.
\section{Verder werk}
De huidige implementatie van een RF, geïmplementeerd in de \texttt{scikit-learn} bibliotheek, doet niet aan lokale regressie in de bladeren. In plaats daarvan bevatten de bladeren een gewoon getal. Mogelijk kan lokale regressie in de bladeren een betere prestatie leveren.
\\\\
Er werd een probleem aangehaald met een RF: het kan niet goed om met ongeziene omstandigheden, zoals wanneer de fiets nog niet is ingesteld door de gebruiker. Er wordt hiervoor volgende oplossingen voorgesteld: 

\begin{itemize}
\item een standaard model voorzien
\item hetgeen wat het RF voorspelt veranderen van de eigenlijke cadans naar een verschil ten opzichte van een ingestelde basis cadans.
\end{itemize}

\noindent Een standaard model voorzien vergt extra werk. Het tweede voorstel daarentegen is makkelijk te implementeren. Trappen op een basis cadans is gelijkaardig met hoe de fiets werkt zonder cadanscontroller (mits er niet op de knoppen geduwt wordt).
\\\\
Voor het deel van conceptuele drift, zou de grootte van het venster of reservoir bepaald kunnen worden aan de hand van gebruikerstesten. Daarnaast is het uitwerken van een profielensysteem ook zeker een interessant topic om verder aan te werken. Een profielensysteem houdt verschillende modellen bij van verschillende personen. Indien iemand gaat fietsen, met een reeds bestaand profiel, zou het profielensysteem het overeenkomstige profiel moeten activeren.

