\chapter{Conclusie}
Eerst werd er een realistische geparametriseerde simulatie opgezet. Die houdt rekening met verschillende lasten. De simulatie zorgt ervoor dat er gemakkelijk en snel data kan gegenereerd worden voor verschillende testen. Ten tweede werden enkele algoritmes besproken en geëvalueerd. De resultaten tonen dat PA niet geschikt is voor dit probleem, aangezien de fout te groot is en het te vaak moet bijleren. DT en RF presteerden hier goed. Ten derde werd nagegaan of de algoritmes kunnen omgaan met een stochastische update-strategie. Wederom presteert PA slecht. Binaire beslissingsbomen, zowel bij DT als RF, met diepte drie presteren hier ook slecht omdat het verschil tussen fietsersmodel en voorspelling vrij groot blijft gedurende de test. Uit deze twee testen werd er besloten dat enkel DT en RF, met diepte vier of dieper, geschikt zijn voor dit probleem. Uiteindelijk werd de keuze gemaakt om enkel verder te werken met een RF. Vervolgens werden twee technieken getest die kunnen omgaan met conceptuele drift, het veranderen van concept over tijd. De drift die hier getest werd was abrupt in deze context, omdat de \textit{freely chosen cadence} van een fietser praktisch niet veranderd over de levensduur van de fiets. De resultaten tonen aan dat kleinere structuren beter presteren dan grotere. Sampling presteert in het algemeen beter dan een statisch schuivend venster. Of omgaan met conceptuele drift noodzakelijk is, hangt af van hoe men de fiets gebruikt. Er werd ten slotte nog een probleem aangehaald met de testen en trainingsdata. In de simulatie werd een “waarheid” (\textit{ground truth}) berekend met behulp van het fietsersmodel. In realiteit bestaat dit niet en zullen updates incrementeel moeten doorgevoerd worden. Dit moet zeker nog uitgewerkt worden. Uiteindelijk moet er ook nog een implementatie voorzien worden die samenwerkt met de fietscontroller van Ellio.