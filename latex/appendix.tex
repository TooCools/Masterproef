\chapter*{Appendix}
\addcontentsline{toc}{chapter}{Appendix}
\section*{IntuEdrive}
\addcontentsline{toc}{section}{IntuEdrive}
Ir. Tomas Keppens werkte als departementshoofd bij Toyota toen hij in 2010 met het IntuEdrive project begon. In het kader van een aantal masterproeven aan de KU Leuven werd het project stap per stap uitgewerkt.. In het academiejaar 2016-2017 werkte ingenieursstudent Jorrit Heidbuchel de aansturing van het intuEdrive CVT systeem uit. Halverwege 2017 was er het eerste prototype. Later dat jaar richtten Tomas Keppens en Jorrit Heidbuchel samen intuEdrive op.
\\\\
IntuEdrive wil mobiliteit duurzamer en efficiënter maken door twee-wielmobiliteit veiliger te maken. Het bedrijf bouwt en ontwikkelt E-bikes die perfect passen in het leven van zijn klanten: makkelijk in gebruik, veilig en betrouwbaar.
\\
Vandaag telt IntuEdrive 4 werknemers. Ellio gaat in het najaar van 2019 voor het eerst in productie en mikt in 2020 op een verkoop van 400 stuks in België.
\begin{figure}[h]
  \centering
  \includegraphics[width=0.5\linewidth]{images/logo_intuedrive.png}
  \caption{Logo IntuEdrive}
  \label{fig:Logo IntuEdrive}
\end{figure}
\newpage
\begin{thebibliography}{9}
\addcontentsline{toc}{section}{Bibliografie}
\bibitem{hardware implementation} 
Heidbuchel, J. (2017).
\textit{Hardware Implementation and Control Strategy of a High Dynamic CVT Transmission for an E-Bike}.
Katholieke Universiteit Leuven,
Departement werktuigkunde.
 
\bibitem{randomforest paper} 
Breiman, L. (2001).
\textit{Random Forests}.
University of California, Berkely.

\bibitem{encoding cyclical features} 
London I. (2016).
\textit{Encoding cyclical continuous features - 24-hour time}. Geraadpleegd op 8 mei, 2019 via https://ianlondon.github.io/blog/encoding-cyclical-features-24hour-time/

\bibitem{preprocessing faq}
Sarle, W. 
\textit{comp.ai.neural-nets FAQ}.
Geraadpleegd op 8 mei, 2019 via http://www.faqs.org/faqs/ai-faq/neural-nets/part1/preamble.html

\bibitem{pa algorithm}
Crammer, Dekel, Keshet, Shalev-Shwartyz, Singer (2006).
\textit{Online Passive-Aggressive Algorithms}.
Hebrew University of Jerusalem, Jerusalem.

\bibitem{adaptive ml}
Loeffel, P. 
\textit{Adaptive machine learning algorithms for data streams subject to concept drifts}. 
Université Pierre et Marie Curie, Paris VI

\bibitem{factors effecting cadence}
Hansen E. en Smith G. (2009).
Factors Affecting Cadence Choice During Submaximal Cycling and Cadence Influence on Performance.
\textit{International Journal of Sports Physiology and Performance,4 (1), }pp. 3-17.

\bibitem{reservoir sampling}
Vitter J. 
\textit{Random Sampling with a Reservoir}. 
Brown University, Rhode Island

\bibitem{biased reservoir sampling}
Aggarwal, C. (2006).
On Biased Reservoir Sampling in the Presence of Stream Evolution.
\textit{VLDB, 32 (1), }pp. 607-618.

\bibitem{cvt wikipedia}
Bijdragers van Wikipedia.
\textit{Continuously variable transmission}.
Geraadpleegd op 8 mei, 2019 via https://en.wikipedia.org/wiki/Continuously\_ variable\_ transmission

\bibitem{vermogen wikipedia}
Bijdragers van Wikipedia.
\textit{Vermogen (natuurkunde)}.
Geraadpleegd op 8 mei, 2019 via https://nl.wikipedia.org/wiki/Vermogen\_\%28natuurkunde\%29

\bibitem{random forest wikipedia}
Bijdragers van Wikipedia.
\textit{Random forest}.
Geraadpleegd op 8 mei, 2019 via https://en.wikipedia.org/wiki/Random\_ forest

\bibitem{binomial wiki}
Bijdragers van Wikipedia.
\textit{Binomial distribution}.
Geraadpleegd op 8 mei, 2019 via https://en.wikipedia.org/wiki/Binomial\_ distribution

\bibitem{geometric wiki}
Bijdragers van Wikipedia.
\textit{Geometric distribution}.
Geraadpleegd op 8 mei, 2019 via https://en.wikipedia.org/wiki/Geometric\_ distribution

\end{thebibliography}
