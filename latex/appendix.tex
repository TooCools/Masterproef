\chapter*{Appendix}
\addcontentsline{toc}{chapter}{Appendix}
\section*{IntuEdrive}
\addcontentsline{toc}{section}{IntuEdrive}
Ir. Tomas Keppens werkte als departementshoofd bij Toyota toen hij in 2010 met het IntuEdrive project begon. In het kader van aan aantal masterproeven aan de KU Leuven werd het project stap per stap uitgewerkt.. In het academiejaar 2016-2017 werkte ingenieursstudent Jorrit Heidbuchel de aansturing van het intuEdrive CVT systeem uit. Halverwege 2017 was er het eerste prototype. Later dat jaar richtten Tomas Keppens en Jorrit Heidbuchel samen intuEdrive op.
\\\\
IntuEdrive wil mobiliteit duurzamer en efficiënter maken door twee-wielmobiliteit veiliger te maken. Het bedrijf bouwt en ontwikkelt E-bikes die perfect passen in het leven van zijn klanten: makkelijk in gebruik, veilig en betrouwbaar.
\\
Vandaag telt IntuEdrive 4 werknemers. CoSaR gaat in het najaar van 2019 voor het eerst in productie en mikt in 2020 op een verkoop van 1000 stuks in België.
\begin{figure}[h]
  \centering
  \includegraphics[width=0.5\linewidth]{images/logo_intuedrive.png}
  \caption{Logo IntuEdrive}
  \label{fig:Logo IntuEdrive}
\end{figure}
\newpage
\begin{thebibliography}{9}
\addcontentsline{toc}{section}{Bibliografie}
\bibitem{hardware implementation} 
Heidbuchel, J. (2017); Hardware Implementation and Control Strategy of a High Dynamic CVT Transmission for an E-Bike; KULEUVEN
 
\bibitem{randomforest paper} 
Breiman, L. (2001); Random Forests; UC BERKELEY

\bibitem{encoding cyclical features} 
London I. (2016); Encoding cyclical continuous features - 24-hour time; https://ianlondon.github.io/blog/encoding-cyclical-features-24hour-time/

\bibitem{preprocessing faq}
Sarle, W. comp.ai.neural-nets FAQ; http://www.faqs.org/faqs/ai-faq/neural-nets/part1/preamble.html

\bibitem{pa algorithm}
Crammer, Dekel, Keshet, Shalev-Shwartyz, Singer (2006); Online Passive-Aggressive Algorithms; Hebrew University of Jerusalem

\bibitem{cvt wikipedia}
https://en.wikipedia.org/wiki/Continuously\_ variable\_ transmission

\bibitem{vermogen wikipedia}
https://nl.wikipedia.org/wiki/Vermogen\_\%28natuurkunde\%29

\bibitem{random forest wikipedia}
https://en.wikipedia.org/wiki/Random\_ forest

\end{thebibliography}
