% ---------- Titelblad Masterproef Faculteit Wetenschappen -----------
% Dit document is opgesteld voor compilatie met pdflatex.  Indien je
% wilt compileren met latex naar dvi/ps, dien je de figuren naar
% (e)ps-formaat om te zetten.
%                           -- december 2012
% -------------------------------------------------------------------
\RequirePackage{fix-cm}
\documentclass[12pt,a4paper,oneside]{book}

% --------------------- In te laden pakketten -----------------------
% Deze kan je eventueel toevoegen aan de pakketten die je al inlaadt
% als je dit titelblad integreert met de rest van thesis.
% -------------------------------------------------------------------
\usepackage{graphicx,xcolor,textpos}
\usepackage{helvet}
\usepackage{tikz}
\usetikzlibrary{fit,shapes,arrows,positioning,calc}


% -------------------- Pagina-instellingen --------------------------
% Indien je deze wijzigt, zal het titelblad ook wijzigen.  Dit dien je
% dan manueel aan te passen.
% --------------------------------------------------------------------

\topmargin -10mm
\textwidth 160truemm
\textheight 240truemm
\oddsidemargin 0mm
\evensidemargin 0mm

% ------------------- textpos-instellingen ---------------------------
% Enkele andere instellingen voor het voorblad.
% --------------------------------------------------------------------

\definecolor{green}{RGB}{172,196,0}
\definecolor{bluetitle}{RGB}{29,141,176}
\definecolor{blueaff}{RGB}{0,0,128}
\definecolor{blueline}{RGB}{82,189,236}
\setlength{\TPHorizModule}{1mm}
\setlength{\TPVertModule}{1mm}

\renewcommand{\contentsname}{Inhoudstafel}
\renewcommand{\listfigurename}{Lijst van figuren}
\begin{document}

% ---------------------- Voorblad ------------------------------------
% Vergeet niet de tekst aan te passen:
% - Titel en, indien van toepassing, ondertitel
%          voor eventuele formules in de titel of ondertitel
%          gebruik je  \form{$...$}
% - Je naam
% - Je (co)promotor, begeleider (indien van toepassing)
% - Je opleiding
% - Het academiejaar
% --------------------------------------------------------------------
\thispagestyle{empty}
\newcommand{\form}[1]{\scalebox{1.087}{\boldmath{#1}}}
\sffamily
%
\begin{textblock}{191}(-24,-11)
\colorbox{green}{\hspace{113mm}\ \parbox[c][18truemm]{100mm}{\textcolor{white}{FACULTEIT COMPUTERWETENSCHAPPEN}}}
\end{textblock}
%
\begin{textblock}{70}(-18,-19)
\textblockcolour{}
\includegraphics*[height=19.8truemm]{LogoKULeuven}
\end{textblock}
%
\begin{textblock}{160}(-6,63)
\textblockcolour{}
\vspace{-\parskip}
\flushleft
\fontsize{40}{42}\selectfont \textcolor{bluetitle}{Real-Time Cadansaanpassing in een Automatische fietstransmissie }\\[1.5mm]
\end{textblock}
%

\begin{textblock}{160}(8,153)
\textblockcolour{}
\vspace{-\parskip}
\flushright
\fontsize{14}{16}\selectfont \textbf{Arno Cools}
\end{textblock}
%
\begin{textblock}{70}(-6,191)
\textblockcolour{}
\vspace{-\parskip}
\flushleft
Promotor:\\Prof. M. Moens\\[-2pt]
\vspace{5mm}
Begeleider:\\
\textsl{Ir. Tomas Keppens\\Ir. Jorrit Heidbuchel\\Ir. Rugen Heidbuchel}\\[-2pt]
\end{textblock}
%
\begin{textblock}{160}(8,191)
\textblockcolour{}
\vspace{-\parskip}
\flushright
Proefschrift ingediend tot het\\[4.5pt]
behalen van de graad van\\[4.5pt]
Master of Toegepase Informatica\\
\end{textblock}
%
\begin{textblock}{160}(8,232)
\textblockcolour{}
\vspace{-\parskip}
\flushright
Academiejaar 2018-2019
\end{textblock}
%
\begin{textblock}{191}(-24,248)
{\color{blueline}\rule{550pt}{5.5pt}}
\end{textblock}
%
\vfill
\newpage

% Als je het titelblad wil integreren met de rest van je thesis,
% kan je hieronder verder.
% ----------------------- Eerste pagina's -------------------------
% Hier kan je inhoudsopgave, voorwoord en dergelijke kwijt.
% -----------------------------------------------------------------
\rmfamily
\frontmatter
\pagenumbering{roman}
\tableofcontents
\setcounter{page}{0}
\chapter{Voorwoord}

A preface

\chapter{Abstract}

An abstract.

\listoffigures
\addcontentsline{toc}{chapter}{Lijst van figuren}

\chapter{Lijst van symbolen en afkortingen}
lijst van symbolen en afkortingen

\mainmatter
\pagenumbering{arabic}

\chapter{Inleiding}
\tikzset{
block/.style = {draw, fill=white, rectangle, minimum height=3em, minimum width=9em},
tmp/.style  = {coordinate}, 
input/.style = {coordinate},
output/.style= {coordinate},
box/.style={draw=gray,dashed,fill opacity = 0,thick,inner sep=5pt},
test/.style = {}
}
\begin{figure}
\begin{tikzpicture}[auto, node distance=2cm,>=latex']
    \node [input, name=fietsinput] (fietsinput) {};
    \node [block, right of=fietsinput,node distance=5cm] (fietser) {Fietser};
    \node [tmp, right of=fietser,node distance=3cm] (tmp4){};
    \node [block, right of=tmp4,node distance=3cm] (fiets) {Fiets};
    \node [tmp, below of=tmp4] (tmp5) {};
    \node [block, below of=fiets] (controller){Controller};
    \node [block, below of=controller] (cadencecontrol) {Cadans Controller};   
    \node [tmp, right of=fiets,node distance=4cm] (tmp1){};    
    \node [output, right of=tmp1,node distance=2cm] (output) {};
    \node [tmp, below of=tmp1] (tmp2){};
    \node [tmp, below of=tmp2] (tmp3){};
    \node [box, fit=(fiets) (controller)] (metafiets){};
    \node [test, above=0.3cm of metafiets] (metafiets2){Metafiets};
    \draw [->] (fietsinput) -- node{$r$} (fietser);
    \draw [->] (fietser) -- node{$u$} (fiets);
    \draw [->] (cadencecontrol) -- node{} (controller);
    \draw [->] (fiets) -- node [name=y] {$y$}(output);   
    \draw [->] (tmp1) |- (tmp2) -- node {} (controller);
    \draw [->] (tmp2) |- (tmp3) -- node {} (cadencecontrol);
    \draw [-] (controller.west) |- (tmp5) -- node {} (tmp4);
\end{tikzpicture}
\caption{Plant diagram van de fiets}
  \label{fig:plantdiagram}
\end{figure}



\newpage
% ----------------------- Achterblad ------------------------------
% Vergeet niet de tekst aan te passen:
% - Afdeling
% - Adres van de afdeling
% - Telefoon en faxnummer
% -----------------------------------------------------------------
\thispagestyle{empty}
\sffamily
%
\begin{textblock}{191}(113,-11)
{\color{blueline}\rule{160pt}{5.5pt}}
\end{textblock}
%
\begin{textblock}{191}(168,-11)
{\color{blueline}\rule{5.5pt}{59pt}}
\end{textblock}
%
\begin{textblock}{183}(-24,-11)
\textblockcolour{}
\flushright
\fontsize{7}{7.5}\selectfont
\textbf{Computerwetenschappen}\\
Celestijnenlaan 200 A bus 2402\\
3000 LEUVEN, BELGI\"{E}\\
tel. + 32 16 32 77 00\\
fax + 32 16 32 79 96\\
www.kuleuven.be\\
\end{textblock}
%
\begin{textblock}{191}(154,-7)
\textblockcolour{}
\includegraphics*[height=16.5truemm]{sedes}
\end{textblock}
%
\begin{textblock}{191}(-20,235)
{\color{bluetitle}\rule{544pt}{55pt}}
\end{textblock}
\end{document}
