% ---------- Titelblad Masterproef Faculteit Wetenschappen -----------
% Dit document is opgesteld voor compilatie met pdflatex.  Indien je
% wilt compileren met latex naar dvi/ps, dien je de figuren naar
% (e)ps-formaat om te zetten.
%                           -- december 2012
% -------------------------------------------------------------------
\RequirePackage{fix-cm}
\documentclass[12pt,a4paper,oneside]{book}

% --------------------- In te laden pakketten -----------------------
% Deze kan je eventueel toevoegen aan de pakketten die je al inlaadt
% als je dit titelblad integreert met de rest van thesis.
% -------------------------------------------------------------------
\usepackage{graphicx,xcolor,textpos}
\usepackage{helvet}
\usepackage{tikz}
\usetikzlibrary{fit,shapes,arrows,positioning,calc}
\usepackage{wrapfig}
\usepackage{chngcntr}
\usepackage{amsmath}
\usepackage{amssymb}
\usepackage{graphicx,caption}
\usepackage{subcaption}
\usepackage[utf8]{inputenc}
\usepackage{soul}
\usepackage{multirow}
\usepackage{afterpage}
\usepackage{array}
\usepackage{ragged2e}
\usepackage{enumitem}

\newcommand\blankpage{%
    \null
    \thispagestyle{empty}%
    \addtocounter{page}{-1}%
    \newpage}
    
\counterwithout{figure}{chapter}
\counterwithout{table}{chapter}
\newcommand\tab[1][1cm]{\hspace*{#1}}
\usepackage[ruled]{algorithm2e}
\makeatletter
\renewcommand{\algorithmcfname}{Algoritme}
\newcommand{\RemoveAlgoNumber}{\renewcommand{\fnum@algocf}{\AlCapSty{\AlCapFnt\algorithmcfname}}}
\newcommand{\RevertAlgoNumber}{\algocf@resetfnum}
\makeatother

\usepackage[acronym]{glossaries}
\newacronym{fcc}{FCC}{freely chosen cadence}
\newacronym{cvt}{CVT}{continu variabele transmissie}
\newacronym{dt}{DT}{decision tree}
\newacronym{pa}{PA}{passive aggressive}
\newacronym{rf}{RF}{random forest}
\newacronym{ma}{MA}{Moving \ Average \ }
\newacronym{es}{ES}{Exponential \ Smoothing \ }

\newglossaryentry{r}{
  name = \ensuremath{r} ,
  description = Input vector fietser,
}
\newglossaryentry{u_cy}{
  name = \ensuremath{u_{cy}} ,
  description = Input vector fiets\,{} geleverd door de fietser,
}
\newglossaryentry{u_contr}{
  name = \ensuremath{u_{contr}} ,
  description = Input vector fietser\,{} geleverd door de controller
}
\newglossaryentry{y}{
  name = \ensuremath{y} ,
  description = Output vector fiets
}
\newglossaryentry{fcc_est}{
  name = \ensuremath{FCC_{est}} ,
  description = Schatting van FCC\,{} geleverd door de cadanscontroller,
}
\newglossaryentry{v_ref}{
  name = \ensuremath{v_{ref}} ,
  description = Referentie snelheid van de fietser,
}
\newglossaryentry{t_cy}{
  name = \ensuremath{T_{cy}} ,
  description = Koppel geleverd door de fietser,
}
\newglossaryentry{t_cy,m}{
  name = \ensuremath{T_{cy,m}} ,
  description = Gemeten koppel geleverd door de fietser,
}
\newglossaryentry{u_c}{
  name = \ensuremath{u_c} ,
  description = De staat van de knop die de cadans aanpast,
}
\newglossaryentry{theta_cr}{
  name = \ensuremath{\theta_{cr}} ,
  description = Hoek van de trapas,
}
\newglossaryentry{v_bike}{
  name = \ensuremath{v_{bike}} ,
  description = Snelheid van de fiets,
}
\newglossaryentry{helling}{
  name = \ensuremath{\alpha} ,
  description = Helling,
}
\newglossaryentry{t_dc}{
  name = \ensuremath{T_{dc}} ,
  description = Het DC koppel van de fietser (gemiddeld koppel),
}
\newglossaryentry{t_dc_max}{
  name = \ensuremath{T_{dc,max}} ,
  description = Maximum DC koppel dat geleverd kan worden,
}
\newglossaryentry{omega_cr}{
  name = \ensuremath{\omega_{cr}} ,
  description = Cadans in rpm,
}
\newglossaryentry{fcc_pred}{
  name = \ensuremath{FCC_{pred}},
  description = Voorspelde FCC,
}
\newglossaryentry{k_cr,r}{
  name = \ensuremath{k_{cr,r}} ,
  description = Verhouding overbrenging trapas-ringwiel,
}
\newglossaryentry{nr}{
  name = \ensuremath{nr} ,
  description = Aantal tanden op het ringwiel,
}
\newglossaryentry{ns}{
  name = \ensuremath{ns} ,
  description = Aantal tanden op het zonnewiel,
}
\newglossaryentry{s}{
  name = \ensuremath{S} ,
  description = Ondersteuningsniveau,
}
\newglossaryentry{f_grav}{
  name = \ensuremath{F_{grav}} ,
  description = Gravitationele last,
}
\newglossaryentry{f_friction}{
  name = \ensuremath{F_{friction}} ,
  description = Wrijvings last,
}
\newglossaryentry{f_aero}{
  name = \ensuremath{F_{aero}} ,
  description = Luchtweerstand,
}
\newglossaryentry{f_load}{
  name = \ensuremath{F_{load}} ,
  description = Totale last,
}
\newglossaryentry{m}{
  name = \ensuremath{m} ,
  description = Totaal gewicht van fiets en fietser,
}
\newglossaryentry{g}{
  name = \ensuremath{g} ,
  description = Gravitationele constante,
}
\newglossaryentry{c_r}{
  name = \ensuremath{c_r} ,
  description = Rolweerstand coëfficiënt,
}
\newglossaryentry{c_d}{
  name = \ensuremath{c_d} ,
  description = Luchtweerstand coëfficiënt,
}
\newglossaryentry{rho_aero}{
  name = \ensuremath{\rho_{aero}} ,
  description = Luchtdichtheid,
}
\newglossaryentry{a_aero}{
  name = \ensuremath{A_{aero}} ,
  description = Frontaal oppervlak fietser,
}
\newglossaryentry{t_mg2}{
  name = \ensuremath{T_{MG2}} ,
  description = Koppel geleverd door motor op het voorwiel,
}
\newglossaryentry{t_rw}{
  name = \ensuremath{T_{rw}} ,
  description = Koppel op het achterwiel,
}
\newglossaryentry{r_w}{
  name = \ensuremath{r_w} ,
  description = Straal van het voor- en achterwiel,
}
\newglossaryentry{p}{
  name = \ensuremath{P} ,
  description = Vermogen,
}
\newglossaryentry{sf}{
  name = \ensuremath{sf} ,
  description = Smoothing factor,
}
\newglossaryentry{f_k}{
  name = \ensuremath{f_k} ,
  description = Factor voor fietsersmodel in functie van het gemiddeld koppel,
}
\newglossaryentry{f_h}{
  name = \ensuremath{f_h} ,
  description = Factor voor fietsersmodel in functie van de helling,
}
\newglossaryentry{f_v}{
  name = \ensuremath{f_v} ,
  description = Factor voor fietsersmodel in functie van de snelheid,
}
\newglossaryentry{k}{
  name = \ensuremath{K} ,
  description = Agressiviteits parameter voor proportionele regelaar,
}

\setlength{\glsdescwidth}{10.5cm}
\newglossarystyle{mystyle}{%
  \glossarystyle{long}%
  \renewenvironment{theglossary}%
     {\begin{longtable}{p{1.5cm}p{\glsdescwidth}}}%
     {\end{longtable}}%
} 
\makeglossaries


% -------------------- Pagina-instellingen --------------------------
% Indien je deze wijzigt, zal het titelblad ook wijzigen.  Dit dien je
% dan manueel aan te passen.
% --------------------------------------------------------------------

\topmargin -10mm
\textwidth 160truemm
\textheight 240truemm
\oddsidemargin 0mm
\evensidemargin 0mm

% ------------------- textpos-instellingen ---------------------------
% Enkele andere instellingen voor het voorblad.
% --------------------------------------------------------------------

\definecolor{green}{RGB}{172,196,0}
\definecolor{bluetitle}{RGB}{29,141,176}
\definecolor{blueaff}{RGB}{0,0,128}
\definecolor{blueline}{RGB}{82,189,236}
\setlength{\TPHorizModule}{1mm}
\setlength{\TPVertModule}{1mm}

\renewcommand{\contentsname}{Inhoudstafel}
\renewcommand{\listfigurename}{Lijst van figuren}
\renewcommand{\listtablename}{Lijst van tabellen}
\DeclareMathOperator*{\argmin}{argmin}
\renewcommand{\chaptername}{Hoofdstuk}
\renewcommand{\figurename}{Figuur}
\renewcommand{\tablename}{Tabel}

\begin{document}

% ---------------------- Voorblad ------------------------------------
% Vergeet niet de tekst aan te passen:
% - Titel en, indien van toepassing, ondertitel
%          voor eventuele formules in de titel of ondertitel
%          gebruik je  \form{$...$}
% - Je naam
% - Je (co)promotor, begeleider (indien van toepassing)
% - Je opleiding
% - Het academiejaar
% --------------------------------------------------------------------
\thispagestyle{empty}
\newcommand{\form}[1]{\scalebox{1.087}{\boldmath{#1}}}
\sffamily
%
\begin{textblock}{191}(-24,-11)
\colorbox{green}{\hspace{113mm}\ \parbox[c][18truemm]{100mm}{\textcolor{white}{FACULTEIT COMPUTERWETENSCHAPPEN}}}
\end{textblock}
%
\begin{textblock}{70}(-18,-19)
\textblockcolour{}
\includegraphics*[height=19.8truemm]{LogoKULeuven}
\end{textblock}
%
\begin{textblock}{160}(-6,63)
\textblockcolour{}
\vspace{-\parskip}
\flushleft
\fontsize{40}{42}\selectfont \textcolor{bluetitle}{Real-time cadansaanpassing in een automatische fietstransmissie }\\[1.5mm]
\end{textblock}
%

\begin{textblock}{160}(8,153)
\textblockcolour{}
\vspace{-\parskip}
\flushright
\fontsize{14}{16}\selectfont \textbf{Arno Cools}
\end{textblock}
%
\begin{textblock}{70}(-6,191)
\textblockcolour{}
\vspace{-\parskip}
\flushleft
Promotor:\\Prof. M.-F. Moens\\[-2pt]
\vspace{5mm}
Begeleiders:\\
Ir. T. Keppens\\Ir. J. Heidbuchel\\Ir. R. Heidbuchel\\[-2pt]
\end{textblock}
%
\begin{textblock}{160}(8,191)
\textblockcolour{}
\vspace{-\parskip}
\flushright
Proefschrift ingediend tot het\\[4.5pt]
behalen van de graad van\\[4.5pt]
Master in de Toegepase Informatica\\
\end{textblock}
%
\begin{textblock}{160}(8,232)
\textblockcolour{}
\vspace{-\parskip}
\flushright
Academiejaar 2018-2019
\end{textblock}
%
\begin{textblock}{191}(-24,248)
{\color{blueline}\rule{550pt}{5.5pt}}
\end{textblock}
%
\vfill
\newpage

% Als je het titelblad wil integreren met de rest van je thesis,
% kan je hieronder verder.
% ----------------------- Eerste pagina's -------------------------
% Hier kan je inhoudsopgave, voorwoord en dergelijke kwijt.
% -----------------------------------------------------------------
\rmfamily
\frontmatter
\afterpage{\blankpage}
\pagenumbering{roman}
\setcounter{page}{1}
\chapter{Voorwoord}
Ik zou graag iedereen bedanken die mij heeft geholpen met deze thesis. De personen waar ik het meeste aan te danken heb, zijn Jorrit en Rugen Heidbuchel. Beiden gaven zeer nuttige feedback en waren altijd beschikbaar. Dank aan Tomas Keppens om enkele jaren geleden het IntuEdrive project op te starten en samen te werken met de KU Leuven. Bedankt Sien Moens om de thesis te proeflezen. En natuurlijk ook een dank u aan iedereen die deze thesis gelezen heeft.

\chapter{Abstract}
Deze thesis gaat over het real time aanpassen van de fietscadans, toegepast op de E-bike van IntuEdrive. Deze tekst is de laatste van een reeks thesissen rond het E-bike concept van Tomas Keppens: een snelle en betrouwbare E-bike die makkelijk in gebruik is. Deze thesis werkt verder op het huidige prototype. Het doel is om de trapsnelheid van de fietser te kunnen voorspellen om zo de versnelling van de fiets automatisch te schakelen. De trapsnelheid is per gebruiker verschillend. Als eerste stap wordt een geparametriseerde simulatie gemaakt, zodat makkelijk veel consistente data gegenereerd kan worden. Vervolgens worden verschillende algoritmes van machinaal leren, het \textit{Passive Aggressive algoritme}, binaire beslissingsbomen en \textit{random forest}, bekeken en geëvalueerd. Er moet rekening gehouden worden met het aantal keer trainen, uitvoeringstijd en accuraatheid. Vervolgens wordt er nagegaan of de algoritmes om kunnen gaan met een stochastische keuze om het model al dan niet te updaten. De mens is immers een onvoorspelbaar wezen. Ten slotte wordt het probleem van conceptuele drift aangehaald. Een fiets kan immers door meerdere personen gebruikt worden. De resultaten tonen aan dat een random forest het best presteert in de verschillende experimenten.
\tableofcontents
\newpage
\listoffigures
\addcontentsline{toc}{chapter}{Lijst van figuren}
\listoftables
\addcontentsline{toc}{chapter}{Lijst van tabellen}
\newpage
\printglossary[type=\acronymtype ,title=Lijst van afkortingen ,style=mystyle]
\addcontentsline{toc}{chapter}{Lijst van afkortingen}
\newpage
\addcontentsline{toc}{chapter}{Lijst van symbolen}
\printglossary[title=Lijst van symbolen,style=mystyle]

\mainmatter
\pagenumbering{arabic}
\RemoveAlgoNumber

\chapter{Probleemstelling}
\section{Mobiliteitsvraagstuk}
De auto is het slachtoffer geworden van zijn eigen succes: we staan meer dan ooit in de file en de $\text{CO}_2$-uitstoot van personenverkeer stijgt jaar na jaar. De Belg neemt al snel de auto voor korte afstanden ($<$ 25 km). In deze auto zit meestal maar één persoon. Het Belgische wagenpark blijft groeien (figuur \ref{fig:wagenpark}). Hier zien we wel een trend ontstaan. Er worden steeds meer elektrische en hybride wagens verkocht, maar die staan natuurlijk net zo goed in de file. Mobiliteit op twee wielen kan hier een oplossing bieden.
\\

\begin{wrapfigure}{R}{0.40\textwidth}
  \centering
  \includegraphics[width=1.1\linewidth]{images/snelheid-veiligheid-tradeoff.png}
  \caption{Snelheid-veiligheid trade-off (bron: IntuEdrive)}
  \label{fig:snelheid-veiligheid trade-off (bron: IntuEdrive)}
\end{wrapfigure}
\noindent Mobiliteit op twee wielen kennen we al lang: fietsen bestaan al sinds de 19de eeuw. Elektrische fietsen hebben het potentieel van deze tweewielers enorm verhoogd: fietsen wordt moeiteloos en stukken sneller. Spijtig genoeg neemt het risico op ongevallen ook toe bij hogere snelheid. Dat komt omdat e-bikes en speed e-bikes precies dezelfde technologie gebruiken als normale fietsen – grote wielen met smalle banden, kettingaandrijving met manuele versnellingen, mechanische handremmen – bij veel hogere snelheden. IntuEdrive noemt dit de snelheid-veiligheid trade-off. De veiligheid kan beperkt worden verhoogd door componenten toe te voegen (bv. Bosch e-bike ABS), maar de functionaliteit van deze systemen blijft beperkt. Er is een meer holistische aanpak nodig. Bovendien bieden elektrische fietsen vandaag nog niet het gebruiksgemak en de betrouwbaarheid die de consument gewend is van zijn wagen.
\\\\
IntuEdrive’s \textit{Ellio} is een snelle elektrische fiets die veiliger is dan de klassieke mechanische fiets, dankzij de innovatieve tweewielaandrijving en elektrische remfunctie. Dit systeem reduceert de stopafstand met 60\% en maakt schakelen overbodig (automatische versnellingen). Het stapt ook af van de onderhoudsintensieve fietscomponenten (ketting, tandwielen, mechanische remmen). Dit maakt Ellio de perfecte e-bike voor woon-werkverkeer: makkelijk, veilig en betrouwbaar.
\\\\
Door automatisch te schakelen zorgt Ellio ervoor dat de fietser in elke situatie precies zo snel trapt als hij of zij wil. Deze gewenste trapsnelheid – of beter trapcadans – varieert van persoon tot persoon en hangt af van omstandigheden zoals helling, tegenwind en rijsnelheid. Omdat deze gewenste cadans niet op voorhand gekend is, schakelt de transmissie momenteel op basis van een vaste wetmatigheid die tijdens testen getuned is om voor zoveel mogelijk gebruikers comfortabel aan te voelen. Wijkt deze wetmatigheid af van de gewenste cadans van een specifieke gebruiker, dan kan deze gebruiker via knoppen op het stuur tijdens het fietsen zijn of haar cadans manueel aanpassen.
\\
\begin{figure}
  \includegraphics[width=\linewidth]{images/wagenpark_belgie.png}
  \caption{Grootte van het voertuigenpark 2014-2018 (bron: statbel.fgov.be)}
  \label{fig:wagenpark}
\end{figure}


\section{Machinaal leren voor geïndividualiseerde cadanscontrole}
Deze thesis werkt verder op het Ellio-prototype van IntuEdrive. Zoals reeds aangehaald schakelt de fiets automatisch. De trapcadans wordt hierdoor stabiel gehouden op een referentiecadans, ook wanneer de fietser harder of zachter trapt. Het doel is om deze referentiecadans te personaliseren door de voor de gebruiker ideale cadans (ook \textit{\gls{fcc}} genoemd) in real time te voorspellen aan de hand van de toestand van de fiets. Op die manier past de trapsnelheid zich niet alleen  aan de omstandigheden aan, maar ook aan de individuele gebruiker. Deze implementatie zal ervoor zorgen dat de fietser minder frequent zelf de trapcadans moet bijstellen. Om de cadans te personaliseren en dynamisch te maken, zal een algoritme van machinaal leren ontwikkeld worden. Dit algoritme krijgt de toestand van de fiets als input binnen. Daarmee wordt de FCC berekent. Wanneer de fietser besluit om de cadans manueel aan te passen, interpreteert het algoritme dit als een signaal om bij te leren. De FCC kwam op dat moment immers niet overeen met de referentiecadans.
\\\\
Om de performantie van het algoritme van machinaal leren te testen zal het volledig systeem fiets-fietser-cadanscontrole gesimuleerd worden. Het fietsmodel wordt geleverd door IntuEdrive en zal geïmplementeerd worden in Python. Vervolgens worden een aantal algoritmes van machinaal leren vergeleken op basis van een aantal vooraf gedefinieerde performance indicatoren. De algoritmes zijn afkomstig uit scikit-learn, een bibliotheek van machinaal leren.
\\\\
De cadanscontrole moet aan verschillende eisen voldoen. Het algoritme moet draaien op een Raspberry Pi, samen met het controleprogramma van de fiets. Door deze beperkte resources moet het algoritme zo efficiënt mogelijk zijn. De voorspellingen moeten bijna in real time berekend worden. Het doel is om aan 10Hz de cadans aan te passen, maar hoe meer voorspellingen per seconde, hoe beter. Tragere voorspellingen kunnen hinderlijk zijn voor het rijgedrag. Ten slotte moet er ook rekening gehouden worden met de veiligheid van de fietser. Opeenvolgende voorspellingen mogen niet te veel van elkaar verschillen, anders zou de fietser erdoor gestoord kunnen worden en zijn concentratie verliezen. Bovendien mag de cadans nooit hoger dan een bepaalde maximum limiet ingesteld worden.
\\\\
De algoritmes worden geëvalueerd op basis van de gemiddelde kwadraten fout tussen de referentiecadans - afkomstig van het algoritme van machinaal leren - en de FCC van de fietser. De FCC is niet precies gekend en wordt in de simulatie bepaald aan de hand van een fietsersmodel. Dit is een functie die de toestand van de fiets en de fietser (rijsnelheid, helling,...) afbeeldt op de FCC. Simpel gezegd is het fietsersmodel een functie met als input de toestand van de fiets en als output een “optimale cadans”. Deze functie is speculatief en kan makkelijk aangepast worden. Op welke basis de fietser precies zijn freely chosen cadence bepaalt is voor dit onderzoek weinig relevant. Het gaat er hier vooral om dat algoritme van machinaal leren het fietsersmodel kan achterhalen.
\\
\begin{center}
Fietsersmodel:\tab fcc = f(snelheid,koppel,vermogen,helling,...)
\end{center}
Het algoritme moet kunnen bijleren met een kleine hoeveelheid data. De gebruiker zal immers niet vaak manuele aanpassingen doen aan de cadans. Te veel data gebruiken kan een negatieve invloed hebben op reeds correcte voorspellingen. Het algoritme moet ook snel bijleren. Elke verandering moet zo snel mogelijk doorgevoerd worden en moet een betekenisvolle impact hebben.
\section{Huidige systeem}
De fiets van intuEdrive gebruikt een elektrische \gls{cvt} die ervoor zorgt dat er naadloos geschakeld kan worden tussen versnellingen, in tegenstelling tot het traditionele ketting-en-tandwiel systeem. Dit oude systeem schakelt in discrete trappen, waardoor de fietser tijdens het schakelen een discontinuïteit voelt. Het CVT-systeem gebruikt 2 motoren en schakelt traploos. Eén van de motoren regelt de trapcadans, de andere motor regelt het ondersteuningsniveau. Het ondersteuningsniveau bepaalt hoeveel extra elektrisch vermogen er geleverd wordt, bovenop wat de fietser zelf levert.
\\\\
Figuur \ref{fig:Blokdiagram van het fiets-fietser-controller systeem} toont een blokdiagram van het systeem fiets-fietser-controller. We gaan ervan uit dat de fietser op elk moment een bepaalde referentiesnelheid (\gls{v_ref}) probeert te halen, hier aangeduid met \gls{r}. Die kan variëren naargelang de situatie, maar is voor elke gebruiker anders. Tijdens het fietsen geeft de fietser input aan de fiets (\gls{u_cy}). Zo kan hij of zij het geleverde koppel variëren (\gls{t_cy}) – i.e. meer of minder kracht op de pedalen zetten – of de cadans aanpassen met de knoppen (\gls{u_c}). Inputs en fysische toestand van de fiets worden gemeten door sensoren op de fiets: het koppel (\gls{t_cy,m}), de hoek van de trapas (\gls{theta_cr}), snelheid (\gls{v_bike}), helling (\gls{helling}), etc. $T_{cy}$ en $T_{cy,m}$ zijn niet hetzelfde, want er kunnen fouten gebeuren tijdens het meten. De vector van meetwaarden (\gls{y}) is input voor de fietscontroller. De fietscontroller stuurt de motoren in de E-bike aan (\gls{u_contr}) op basis van de metingen $y$ en de ingestelde referentiecadans. De cadanscontroller die in deze thesis uitgewerkt zal worden, zal op basis van dezelfde metingen een gepersonaliseerde referentiecadans (\gls{fcc_est}) voorspellen die als input dient voor de controller.
\tikzset{
block/.style = {draw, fill=white, rectangle, minimum height=3em, minimum width=9em},
smallblock/.style = {draw, fill=white, rectangle, minimum height=3em, minimum width=3em},
tmp/.style  = {coordinate}, 
input/.style = {coordinate},
output/.style= {coordinate},
box/.style={draw=gray,dashed,fill opacity = 0,thick,inner sep=5pt},
test/.style = {}
}
\begin{gather*}
r = \begin{bmatrix}
       v_{ref}  
     \end{bmatrix} \tab
u_{cy} = \begin{bmatrix}
       T_{cy} \\ u_c  
     \end{bmatrix} \tab
cc = \begin{bmatrix}
       FCC_{est}  
     \end{bmatrix} \tab
y = \begin{bmatrix} 
       \theta _{cr} \\ T_{cy,m} \\ v_{bike} \\ \alpha
     \end{bmatrix} 
\end{gather*}
\begin{figure}[h]
\begin{tikzpicture}[auto, node distance=2cm,>=latex']
    \node [input, name=fietsinput] (fietsinput) {};
    \node [block, right of=fietsinput,node distance=5cm] (fietser) {Fietser};
    \node [tmp, right of=fietser,node distance=3cm] (above_fietser){};
    \node [block, below of=above_fietser,node distance=3cm] (fiets) {Fiets};
    \node [tmp, left = 1.5cm of fiets] (left_fiets) {};
    \node [block, below of=fiets] (controller){Controller};
    \node [tmp, left = 1.5cm of controller] (left_controller) {};
    \node [block, below of=controller] (cadencecontroller) {Cadanscontroller};   
    \node [tmp, left = 1.5cm of cadencecontroller] (left_cadencecontrol) {};
    \node [tmp, right of=fiets,node distance=3cm] (right_fiets){};    
    \node [tmp, right of=right_fiets] (output){};    
    \draw [->] (fietsinput) -- node{$r$} (fietser);
    \draw [->] (fietser) |- (above_fietser) -- node{$u_{cy}$} (fiets);
    \draw [->] (controller.west) |- (left_controller) |- node {$u_{contr}$} (fiets.west);
    \draw [->] (cadencecontroller) -- node{$cc$} (controller);
   	\draw [->] (right_fiets) |- (controller.east);
   	\draw [->] (right_fiets) |- (cadencecontroller.east);
    \draw [->] (fiets) -- node [name=y] {$y$}(output);   
\end{tikzpicture}
\caption{Blokdiagram van het fiets-fietser-controller systeem}
  \label{fig:Blokdiagram van het fiets-fietser-controller systeem}
\end{figure}
\newpage
\section{Gerelateerd werk}
\subsection{Hardware Implementation and Control Strategy of a High Dynamic CVT Transmission for an E-Bike: Jorrit Heidbuchel (2016-2017)}
Jorrit werkte in zijn masterproef aan het verbeteren van de traditionele elektrische fiets. Elektrische fietsen gebruiken een elektrische motor om extra vermogen toe te voegen, bovenop het vermogen dat de fietser zelf levert. De klassieke fietssystemen die op E-bikes gebruikt worden, hebben hun beperkingen. Ten eerste kan er enkel geremd worden met behulp van mechanische remmen. Dit remsysteem is gevoelig voor wielslip en laat niet toe om remenergie te recupereren.  Daarnaast schakelt de traditionele elektrische fiets in discrete stappen. Om deze tekortkomingen op te lossen, werkte  Jorrit een continu variabele transmissie uit die ook toelaat om op de elektrische motoren af te remmen. Zo kan er volledig automatisch en traploos geschakeld worden en kan bovendien elektrische energie worden gerecupereerd tijdens een remmanoeuvre. 
\\\\
Jorrit’s masterproef werkte verder op thesissen in het kader van ir. Tomas Keppens’ concept voor een dergelijke aandrijving. Die aandrijving was uitgebreid gesimuleerd en moest worden uitgewerkt tot een eerste prototype. Het eerste IntuEdrive prototype was het resultaat van Jorrits masterproef.


\subsection{Factors affecting cadence choice during submaximal cycling and cadence influence on performance: Ernst A. Hansen en Gerald Smith (2009)}
Hansen en Smith bestudeerden de factoren die de cadans keuze van een fietser beïnvloeden. Factoren zoals hellingsgraad, leeftijd, fietsvermogen, duur en vele anderen blijken invloed te hebben op de freely chosen cadance van een fietser.
\\\\
Hansen en Smith introduceren de termen \textit{freely chosen cadence (FCC)} en \textit{energetically optimal cadence (EOC)}. De FCC is de cadans die de fietser zelf kiest als meest comfortabele omwentelingssnelheid.. De EOC is de cadans waarbij de zuurstofopname optimaal is. Tijdens fietsen op lage intensiteit, kiezen fietsers een cadans die hoger is dan de energie optimale cadans en dus fietsen ze minder energie efficiënt. Tijdens fietsen op hoge intensiteit, kiezen fietsers een cadans die dichter ligt bij de energie optimale cadans wat leidt tot betere prestaties. De mens fietst dus niet altijd op een energie-efficiënte manier. Dit is een verschil met het stapgedrag van de mens. De gekozen stapcadans (\textit{freely chosen step cadence}) is wel energie optimaal. Vanuit een evolutionair standpunt is dit logisch. We stappen en lopen al duizenden jaren. De fiets is in dit opzicht nog een recente uitvinding.

\subsection{Adaptive machine learning algorithms for data streams subject to concept drifts: Pierre-Xavier Loeffel (2018)}
Loeffel haalt in zijn thesis verschillende manieren aan om om te gaan met conceptuele drift in data streams. Conceptuele drift is de verandering van het concept na verloop van tijd. Loeffel geeft de voor- en nadelen van verschillende technieken zoals: \textit{sliding window, sampling} en \textit{fading factor}. Alle technieken implementeren een vorm van “vergeten” en zijn mogelijk blind of geïnformeerd. Sliding window en sampling zijn vormen van abrupt vergeten en fading factor is een vorm van geleidelijk vergeten.
\\\\
Naast de reeds aangehaalde technieken, introduceert Loeffel een nieuw algoritme, “the \gls{dea}”, voor classificatieproblemen. DEA, in tegenstelling tot andere ensemble algoritmes, leert de expertise-regio’s van de onderliggende algoritmes van machinaal leren. Het selecteert dynamisch een subset van deze onderliggende algoritmes om een betere voorspelling te maken. Zo kunnen betere voorspellingen gemaakt worden.

\chapter{Methode}
\section{De fietssimulatie}
Er wordt een simulatie gemaakt die de toestand van de fiets zo goed mogelijk probeert te benaderen. Er zal geen rekening gehouden worden met het manoeuvreren van de fiets of van tegenwind. Enkel de relevante meetwaarden worden bijgehouden. De simulatie is geparametriseerd om eenvoudig verschillende scenario’s te testen.
\\\\
Het voordeel van de fietssimulatie is de enorme flexibiliteit. Uren aan data kunnen in een moment tijd gegenereerd worden, waardoor het makkelijk is om verschillende tests uit te voeren. Hiervoor moeten slechts enkele instellingen aangepast worden. Het is ook mogelijk om slechts een enkele parameter aan te passen tijdens tests terwijl de rest constant blijft (\textit{ceteris paribus}), wat praktisch onmogelijk is in een veldtest. Omdat de simulatie bovendien een duidelijke referentie genereert voor de FCC (output van het fietsersmodel), kan de performantie van de cadanscontroller op een kwantitatieve manier worden geëvalueerd. Tijdens een veldtest zou de fietser alleen kwalitatief kunnen aangeven of hij of zij de voorspelde cadans goed vind. 
\section{Modelleren van het fietserkoppel}
Het fietserkoppel wordt gemodelleerd als een sinusfunctie met twee pieken per omwenteling van de trapas (2 benen), met het DC koppel van de fietser als parameter.
\\
\begin{gather*}
 T_{cy} = \gls{t_dc}(1+sin(2\theta_{cr}-\frac{\pi}{6}))
\end{gather*}
\\
Uit deze formule is het ook meteen duidelijk dat het DC koppel ook het vermogen-equivalent koppel is. Dat wil zeggen dat het DC koppel gedurende een volledige omwenteling van de trapas evenveel arbeid levert als het fietserkoppel.
\\
\begin{align*}
\int_{0}^{2\pi} T_{cy}(1+sin(2x-\frac{\pi}{6})) dx &= T_{cy} \int_{0}^{2\pi}(1+sin(2x-\frac{\pi}{6})) dx\\
&= T_{cy} \left[x-\frac{1}{2}sin(\frac{1}{3}(6x+\pi))\right]_0^{2\pi}\\
&= 2\pi \ T_{cy}
\end{align*}
\\
\noindent
Het gemiddelde koppel geleverd door de fietser wordt gemodelleerd als een proportionele regelaar. Het doel is om een bepaalde snelheid, $v_{ref}$, te behalen. Hoe groter het verschil is tussen de referentie snelheid en de eigenlijke snelheid, hoe meer kracht er geleverd zal worden. Als deze referentie snelheid overschreden wordt, dan zal er geen koppel meer geleverd worden. Dit wordt ook wel freewheelen genoemd. Om de kracht van de actor te limiteren, wordt er een maximum koppel ingesteld ($T_{dc,max}$) naar gelang de huidige cadans (\gls{omega_cr}). Zo wordt er meer kracht geleverd wanneer de cadans laag is, net zoals in de werkelijkheid. \gls{k} bepaalt de agressiviteit van de regelaar. De formules zien er als volgt uit:

\begin{gather*}
\gls{t_dc_max} = \frac{-\omega_{cr}}{2}+60 \tab (Figuur\ \ref{fig:koppeltoerentalkarakteristiek}) \\
T_{dc} = min(T_{dc,max},max(0,-K*(v_{bike}-v_{ref}))
\end{gather*}
\\
\begin{figure}
  \includegraphics[width=\linewidth]{images/koppel-toerentalkarakteristiek.png}
  \caption{Het koppel-toerentalkarakteristiek}
  \label{fig:koppeltoerentalkarakteristiek}
\end{figure}
\\
\noindent Figuren \ref{fig:menselijkkoppelverloop} en \ref{fig:gesimuleerdkoppelverloop} tonen een menselijk koppelverloop en gesimuleerd koppelverloop, gesampled aan 10Hz. Zoals te zien is het gesimuleerde koppel heel consistent. Het menselijk koppel volgt duidelijk een cyclische functie, maar toont vormen van inconsistentie. Merk wel op dat er telkens een afwisseling is van een hoge en een lage piek. Dit wijst op een dominant been. Figuur \ref{fig:gesimuleerde koppel dominant been} toont een gesimuleerd koppelverloop van een fietser met een dominant been.
\\\\
\begin{figure}[t!]
\centering
\begin{subfigure}{.5\textwidth}
  \centering
  \includegraphics[width=\linewidth]{images/menselijkkoppel.png}
  \caption{Menselijk koppelverloop}
  \label{fig:menselijkkoppelverloop}
\end{subfigure}%
\begin{subfigure}{.5\textwidth}
  \centering
  \includegraphics[width=\linewidth]{images/gesimuleerdekoppel.png}
  \caption{Gesimuleerd koppelverloop}
  \label{fig:gesimuleerdkoppelverloop}
\end{subfigure}
\begin{subfigure}{.5\textwidth}
  \centering
  \includegraphics[width=\linewidth]{images/gesimuleerdekoppeldominantbeen.png}
  \caption{Gesimuleerd koppelverloop met dominant been}
  \label{fig:gesimuleerde koppel dominant been}
\end{subfigure}
\caption{Het koppelverloop van een mens (linksboven), de simulatie (rechtsboven) en een gesimuleerd dominant been (onderaan)}
\label{fig:koppelverloop mens-simulatie}
\end{figure}
\newpage
\begin{wrapfigure}{R}{0.5\textwidth}
  \centering
  \includegraphics[width=\linewidth]{images/planeetwielmechanisme.png}
  \caption{Planeetwielmechanisme (bron: wikipedia)}
  \label{fig:planeetwielmechanisme}
\end{wrapfigure}

\noindent $T_{cy}$ is het koppel op de trapas. Dit moet nog overgebracht worden op het achterwiel. CoSaR maakt gebruik van een planeetwielmechanisme (figuur \ref{fig:planeetwielmechanisme}). Dit mechanisme laat toe om een grote overbrengingsverhouding te voorzien in een kleine ruimte. Het achterwiel-koppel wordt beïnvloed door het aantal tanden op het zonnewiel (1; \gls{ns}) en het ringwiel (2; \gls{nr}) en de overbrengingsverhouding tussen de trapas en het ringwiel (\gls{k_cr,r}). Het koppel op het achterwiel (\gls{t_rw}) ziet er als volgt uit:
\begin{gather*}
T_{rw}=T_{cy}*k_{cr,r}*\frac{nr+ns}{nr}
\end{gather*}
\\\\
Bovenop het vermogen geproduceerd door de fietser, levert CoSaR extra ondersteuning a.d.h.v. een motor (\gls{t_mg2}) gekoppeld aan het voorwiel. De fietser kan zelf een ondersteuningsniveau (\gls{s}) instellen tussen 0 en 5. Hoe hoger dit ondersteuningsniveau, hoe minder inspanning de fietser moet leveren. 
\begin{gather*}
T_{MG2}=min(35,S\thinspace .\thinspace T_{cy})
\end{gather*}

\section{Het fietsersmodel}
Hoe kiest een fietser zijn cadans? Dit is voor elke fietser verschillend en er is nog nauwelijks onderzoek naar gebeurd. Wielrenners trainen om sneller te kunnen trappen omdat dit efficiënter is. Ze kunnen een gemiddeld vermogen leveren van 300 Watt. De doorsnee fietser levert gemiddeld ongeveer 75 Watt tijdens een normale fietstocht. Het fietsersmodel zal hierop worden afgesteld, aangezien wielrenners niet de voornaamste doelgroep zijn voor CoSaR.
\\\\
Het fietsersmodel is een functie die op verschillende manieren uitgedrukt kan worden: op basis van de helling, gemiddeld koppel, of snelheid. Wat het correcte model is wordt in deze thesis niet uitgewerkt. Het is vooral van belang dat de cadanscontroller het model zo snel en zo nauwkeurig mogelijk kan achterhalen, ongeacht wat het model precies is. Hier wordt de volgende aanname gemaakt: hoe hoger het koppel geleverd door de fietser, hoe hoger de gewenste cadans. Wanneer de fietser bijvoorbeeld een helling oprijdt schakelt hij of zij een versnelling omlaag zodat de kracht die op de pedalen gezet moet worden aangenaam blijft. We stellen hier volgende eenvoudige modellen voor:
\begin{align*}
Gemiddeld \ koppel:\tab fcc &= \gls{f_k} . T_{dc}\\
Helling:\tab fcc &= \gls{f_h} . \alpha\\
Snelheid:\tab fcc &= \gls{f_v} . v_{bike}
\end{align*}
Er wordt verder aangenomen dat de fietser ook een zeker lineariteit verwacht bij lage snelheden. Dat wil zeggen dat een fietser het niet comfortabel vindt wanneer hij of zij snel moet trappen wanneer de fiets nog stilstaat of heel traag rijdt, ook al moet er op dat moment veel koppel geleverd worden om te kunnen vertrekken. Daarom wordt bij lage snelheden de FCC begrensd door een lineair oplopend maximum, te vergelijken met een mechanische fietsversnelling. Omdat de doorsnee fietser niet heel traag of heel snel trapt wordt de FCC begrensd tussen de 40 en 120 rpm.
\begin{figure}
  \includegraphics[width=\linewidth]{images/cadansverloop.png}
  \caption{Verwacht cadansverloop in functie van de snelheid.}
  \label{fig:cadansverloop}
\end{figure}
\newpage
\section{Het lastmodel}
De simulatie is voorzien van een lastmodel. Zoals in realiteit, werken lasten in op de fiets. Zwaartekracht, wrijving met de weg en luchtweerstand zijn gemodelleerd als volgt:
\\
\begin{align*}
\gls{f_grav}&=\gls{m} \ . \ \gls{g} \ . \ sin \ \alpha \\
\gls{f_friction}&=m \ . \ g \ . \ \gls{c_r} \ . \ cos \ \alpha \\
\gls{f_aero}&=\frac{\gls{c_d} \ . \ \gls{rho_aero} \ . \ \gls{a_aero} \ . \ v_{bike}^2}{2}
\end{align*}
Samen vormen ze de totale belasting op de fiets.
\[\gls{f_load} = F_{grav}+F_{friction}+F_{aero}\]
Deze lasten zorgen ervoor dat de simulatie een realistische hoeveelheid vermogen nodig heeft om een bepaalde snelheid te halen. Er wordt hier geen rekening gehouden met de wind. Ten eerste zou dit extra complexiteit toevoegen aan de simulatie. En ten tweede vermoeden we volgende hypothese:
\\\\
\tab De freely chosen cadence hangt af van de hoeveelheid last, van welke bron dan \tab ook, die de gebruiker ondervindt en de gebruiker zelf.
\\\\
Het voorgestelde lastmodel omvat deze vereiste. Door de helling en referentie snelheid te variëren ondergaat de fietser een veranderende last. Zoals in de realiteit zoeken mensen een bepaalde snelheid te halen. Wanneer de fietser een te hoge last ondervindt, bijvoorbeeld door een berg op te rijden, moet hij of zij meer vermogen genereren om zijn of haar gewenste snelheid te behouden. Hiervoor zijn 2 mogelijkheden: het verhogen van het koppel of de trapsnelheid. Mensen zijn meer geneigd om hun gewenste cadans te behouden, ongeacht het koppel (binnen bepaalde grenzen). De formule voor mechanisch vermogen gaat als volgt:
\[\gls{p}=T_{cy} \ . \omega_{cr} \]
Om het lastmodel correct te laten werken, moet er nog een helling gegenereerd worden. Om veel werk uit te sparen met het uitstippelen van parcours, wordt dit dynamisch gegenereerd met behulp van perlin noise. Perlin noise kan gebruikt worden om willekeurige getallen te genereren waarbij opeenvolgende getallen weinig van elkaar verschillen. Een perfecte kandidaat dus om terrein te simuleren. Om verschillende trajecten te creëren, kan de seed variabele aangepast worden. Een hellingsgraad wordt in de fietswereld vaak percentueel voorgesteld. Deze implementatie heeft echter radialen nodig. De helling zal beperkt worden tussen $\approx$ 0 en 10\% (0 en 0.1 radialen). Ter vergelijking, de Koppenberg heeft een gemiddeld stijgingspercentage van 11.6\%. Het minimum stijgingspercentage is zo gekozen dat de simulatie zo weinig mogelijk gaat freewheelen.
\begin{figure}[t]
  \includegraphics[width=\linewidth]{images/parcour_slope_example.png}
  \caption{Voorbeeld helling verloop}
  \label{fig:hellingverloop}
\end{figure}
\newpage
\section{Snelheidsvergelijking}
\noindent De snelheid wordt berekend met een standaardformule: vorige snelheid plus acceleratie met respect tot de genomen tijdsprong. De acceleratie is in functie van de last, het totaalgewicht (m), het vermogen geleverd door de fietser op het achterwiel ($T_{rw}$) en het vermogen van een motor bevestigd op het voorwiel ($T_{MG2}$). 
\\\\
De bewegingsvergelijking van de fiets is, met inbegrip van het lastmodel en het fietsersmodel:
\[F \ = \  m \ . \ a \]
Deze vergelijking wordt elke tijdsstap geïntegreerd met behulp van een voorwaartse Euler methode:
\[F = m.(\frac{v_{bike}[h]-v_{bike}[h-1]}{\Delta t})\]
\[ \frac{\Delta t. F}{m}=v_{bike}[h]-v_{bike}[h-1]\]
\[v_{bike}[h]=v_{bike}[h-1]+\Delta t .\frac{1}{m}.F\]
\[v_{bike}[h] \ = \ v_{bike}[h-1] \ + \Delta t  \ . \frac{1}{m} \ . \ (\frac{T_{MG2} \ + \ T_{rw}}{r_w} \ - \ F_{load})\]
De volledige simulatie ziet er als volgt uit:
\\\\
 \fbox{\begin{minipage}{\linewidth}
for h in 1..$\#$tijdssprongen\\
\tab $T_{dc,max} = \frac{-\omega_{cr}[h-1]}{2}+60$\\
\tab $T_{dc} = min(T_{dc,max}, \ max(0,-K*(v_{bike}[h-1]-v_{ref}))$\\
\tab $fcc = f(T_{dc})$\\
\tab $\omega_{cr}=cadans(v_{bike}[h-1], \ T_{dc}, \ fcc)$\\
\tab $\theta_{cr}=\theta_{cr}[h-1] + \Delta t \ . \ \omega_{cr}$ \\
\tab $T_{cy} = T_{dc}(1+sin(2\theta_{cr}-\frac{\pi}{6}))$\\
\tab $T_{rw}=T_{cy}*k_{cr,r}*\frac{nr+ns}{nr}$\\
\tab $T_{MG2}=min(35, \ S \ . \ T_{cy})$\\
\tab $F_{grav}=m \ . \ g \ . \ sin \ \alpha$\\
\tab $F_{friction}=m \ . \ g. \ c_r \ . \ cos \ \alpha$\\
\tab $F_{aero}=\frac{c_d \ . \ \rho_{aero} \ . \ A_{aero} \ . \ v_{bike}[h-1]^2}{2}$\\
\tab $F_{load} = F_{grav}+F_{friction}+F_{aero}$\\
\tab $v_{bike} \ = \ v_{bike}[h-1] \ + \Delta t  \ . \frac{1}{m} \ . \ (\frac{T_{MG2} \ + \ T_{rw}}{\gls{r_w}} \ - \ F_{load})$
\end{minipage}}

\section{On-line voorspellen van de cadans}
Een groot deel van de toestand van de fiets wordt berekend. Om een efficiënt algoritme te creëren moet enkel de relevante data bekeken worden. Zo worden de volgende attributen gebruikt: snelheid, koppel, hoek van de trapas en helling. 
\\

\begin{wrapfigure}{R}{0.40\textwidth}
  \centering
  \includegraphics[width=\linewidth]{images/trapcyclus.png}
  \caption{Evolutie koppel in functie van hoek trapas}
  \label{fig:Evolutie koppel in functie van hoek trapas}
\end{wrapfigure}
\noindent De helling is vanzelfsprekend. Dit is de voornaamste vorm van last en zal dus een impact hebben op de freely chosen cadence van de fietser. De hoek van de trapas op zich heeft niet veel betekenis. En enkel het koppel ook niet. Er kan bijvoorbeeld een koppel geleverd worden van 20Nm. Dit koppel kan op verschillende plaatsen geleverd worden in de trapcyclus. Als dit in de situatie 2 geleverd wordt, dan is dit waarschijnlijk het laagste koppel in de trapcyclus. Wordt dit geleverd gedurende de neergaande beweging (1), dan is dit het hoogste koppel. Deze verschillende situaties zullen een verschillende cadans nodig hebben. Tijdens de eerst situatie wordt er gemiddeld meer koppel geleverd, wat wijst op een grote last. Dus verwachten we hier een hoge cadans. De tweede situatie daarentegen zal gemiddeld een lagere last hebben. Daarom wordt in conjunctie met het koppel de hoek van de trapas gebruikt. Snelheid is ook een relevant attribuut. In situaties met verschillende snelheden en dezelfde last gaat de cadans verschillen.
\newpage
\section{Preprocessing}
Niet elk algoritme heeft nood aan genormaliseerde input. Voor algoritmes die een afstandsfunctie gebruiken is standaardisatie cruciaal. De doelvariabele daarentegen moet niet genormaliseerd worden volgens Warren S. Sarle \cite{preprocessing faq}. In zijn FAQ schrijft hij dat het standaardiseren van de output voornamelijk voordelige effecten heeft op de initiële gewichten. Wanneer er meerdere doelvariabele zijn en als deze ver uit elkaar liggen, kan het wel nuttig zijn om deze te normaliseren. In dit geval is dat niet nodig aangezien enkel de cadans voorspeld zal worden.
\\\\
\noindent Ten eerste zal de data in sequentie gegoten worden. Een sequentie wordt gezien als een aantal vectoren ($x_t$) over verschillende tijdstippen. Een enkele data meting op zich is niet genoeg om een accurate voorspelling te maken, maar te veel data gebruiken is ook niet goed aangezien de voorspellingen tijdig moeten geleverd worden.
\\
\begin{gather*}
x_t = \begin{bmatrix} 
       \theta _{cr} \\ T_{cy,m} \\ v_{bike} \\ \alpha
     \end{bmatrix} \tab
sequentie = \begin{bmatrix} 
       x_t \\ x_{t-1} \\ ... \\ x_{t-n}
     \end{bmatrix} 
\end{gather*}
\\
\noindent Ten tweede zal de hoek van de trapas niet in zijn zuivere vorm gebruikt worden. De hoek van de trapas is een variabele tussen nul en twee pi. Het probleem hier is dat het begin en het einde van een cyclus ver uit elkaar liggen. Voor de mens is het evident dat nul en twee pi hetzelfde zijn, maar voor de computer is dit een groot verschil. Daarom zal de sinus en de cosinus van de hoek genomen worden zodat het begin en einde dicht bij elkaar liggen (figuur \ref{fig:preprocessing hoek trapas}). Zo leren we het algoritme bij dat de data zich cyclisch gedraagt.

\begin{figure}[h]
  \centering
  \includegraphics[width=\linewidth]{images/preprocessing-hoek.png}
  \caption{Figuur links toont dat het begin en einde van een trapcyclus ver uit elkaar liggen. Figuur rechts toont dat beide punten van de linkse figuur dicht bij elkaar liggen.}
  \label{fig:preprocessing hoek trapas}
\end{figure}

\noindent Ten slotte kan er ruis zitten op de metingen. Er zal altijd wel een klein foutje zitten op de data omdat de meetapparatuur niet perfect is. Het is mogelijk dat trillingen van de motor of het wegdek een impact kunnen hebben, voornamelijk op het geleverde koppel. Voor deze iteratie zal hier geen rekening mee gehouden worden. Ruis van het wegdek komt voor op een frequentie van ongeveer 20 Hz. Dit kan nog opgemerkt worden als er op voorhand data wordt gesampled op hogere frequentie. Ruis van de motoren daarentegen komt voor op 13000 Hz. Ver boven de gewilde sample frequentie en zal dus afgebeeld worden op lagere frequenties. Het huidige systeem zal geen rekening houden met beide soorten ruis. 

\begin{figure}[h]
  \centering
  \includegraphics[width=\linewidth]{images/fft_fietser.png}
  \caption{Een Fast Fourier Transformatie van het menselijk koppelverloop.}
  \label{fig:preprocessing hoek trapas}
\end{figure}

\section{Algoritmes}
\subsection{Passive Aggressive Algorithm}
Het \gls{pa} algoritme, beschreven in de paper van Crammer et al. \cite{pa algorithm}, is een online algoritme gelijkaardig aan een perceptron. Net zoals de perceptron, doet PA de matrixvermenigvuldiging $y_t=w_t \cdot x_t$ om de voorspelling te berekenen. Het grootste verschil tussen beide is hoe de gewichten geüpdatet worden. Het PA algoritme is bruikbaar voor classificatie, regressie, uniclass voorspellingen en multiclass problemen.
\\\\
\noindent Het PA algoritme, zoals de naam weggeeft, kan zich zowel passief als agressief gedragen. Het trainen van PA bestaat uit twee stappen. In de eerste stap wordt er een voorspelling $y_{t,p}$ gemaakt a.d.h.v. de input vector $x_t$ en gewichtenmatrix $w$. Hierna wordt het echte label $y_t$ bekendgemaakt. Als de fout kleiner is dan een voorgedefinieerde waarde $\epsilon$, dan zullen de gewichten niet geüpdatet worden. Als de error toch groter is dan deze marge, dan zullen de gewichten $w$ aangepast worden zodat de fout voor de huidige instantie nul wordt. PA past de gewichten aan zodat het verschil tussen het vorige gewicht en het nieuwe gewicht minimaal is. 
\[
    loss_{\epsilon} (w_t;(x_t,y_t))=\left\{
                \begin{array}{ll}
                  0 \tab \tab \tab \ \ |w \cdot x - y| \leq \epsilon \\
                  |w \cdot x - y| - \epsilon \tab anderzijds
                \end{array}
              \right.\\
\]
\[
    w_{t+1}= \argmin_{w \in \mathbb{R}^n}
     \frac{1}{2}||w-w_t||^2 \tab zodat \ loss_{\epsilon} (w_{t+1};(x_t,y_t)) = 0
\]
Door de agressiviteit van het standaard PA algoritme kunnen er problemen ontstaan wanneer er veel ruis zit op de data. Daarom heeft Crammer et al. twee extra versies, PA-I en PA-II, gemaakt die dit probleem oplost. Beide versies voegen een slack variabele $\xi$ toe. Deze variabele zorgt ervoor dat bij het aanpassen van de gewichten, de fout kleiner of gelijk moet zijn aan $\xi$ in plaats van nul. Beide versies hebben ook een agressiviteits parameter C die deze $\xi$ beïnvloedt. Dit is een vorm van regularisatie om overfitting te voorkomen.

\[
   PA-I \tab w_{t+1}= \argmin_{w \in \mathbb{R}^n}
     \frac{1}{2}||w-w_t||^2 + C\xi \tab zodat \ loss_{\epsilon} (w_{t+1};(x_t,y_t)) \leq \xi
\]
\[
   PA-II \tab w_{t+1}= \argmin_{w \in \mathbb{R}^n}
     \frac{1}{2}||w-w_t||^2 + C\xi^2 \tab zodat \ loss_{\epsilon} (w_{t+1};(x_t,y_t)) \leq \xi
\]

De nieuwe gewichtenmatrix $w_{t+1}$ wordt als volgt berekent:
\[
w_{t+1}=w_t+ \tau_t y_t x_t
\]
\begin{align*}
\tau_t &= \frac{loss_t}{||x_t||^2} \tab \tab (PA)\\
\tau_t &= min \{ C, \frac{loss_t}{||x_t||^2} \} \ \ (PA-I)\\
\tau_t &= \frac{loss_t}{||x_t||^2+\frac{1}{2C}} \tab (PA-II) 
\end{align*}


\subsection{Decision Tree en Random Forest}
Een \gls{dt} is een rule-based model. Dit algoritme is snel (greedy), maar kan niet goed om met ruis. Dit model is gekend om makkelijk te overfitten. Daarom wordt het \gls{rf} algoritme simultaan bekeken.
\\\\
Een DT is een binaire boom. In elke knoop wordt een binair keuzepunt gemaakt op basis van een attribuut. Dit keuzepunt is gekozen zodat de data optimaal gesplitst is over beide takken. Sci-kit learn biedt de mogelijkheid aan om de maximum diepte van de boom te beperken. Wanneer deze parameter niet ingesteld is, zal de DT blijven groeien totdat alle blad nodes "puur" zijn. Een correcte diepte kiezen is een enorm moeilijke taak.
\\\\ 
RF is een ensemble. Dit wilt zeggen dat meerdere algoritmes, in dit geval meerdere DT’s, gebruikt worden om een betere voorspelling te maken. L. Breiman \cite{randomforest paper} beweert is zijn paper dat alle RF’s convergeren zodat overfitting geen probleem is. In tegenstelling tot DT’s, kiest een RF geen optimaal attribuut wanneer een node gesplitst wordt. De variabele worden at random gekozen, waardoor geen enkele boom dezelfde is. De verschillende bomen trainen niet met exact dezelfde trainingsdata. Ze passen Bootstrap Aggregating toe, of bagging. Dit houdt in dat uit de originele trainingsset data wordt gesampled at random. Een instantie kan meerdere keren gesampled worden. Deze techniek reduceert de variantie.
\section{Postprocessing}
De cadans variëert lichtjes doorheen een trapcyclus, zowel in het echt als in de simulatie. De voorspellingen zullen dezelfde trend vertonen. Bovendien kan een beetje ruis of het gedrag van de fietser voor afwijkingen zorgen, bijvoorbeeld een grote sprong tussen twee voorspellingen. Dit kan ergerend zijn voor de fietser.
\\\\
Dit probleem kan op verschillende manieren opgelost worden. Gebruikmakend van de huidige en vorige voorspellingen (\gls{fcc_pred}) of de vorige schatting ($FCC_{est}$). De vorige instelling is de cadans die is doorgegeven aan de fiets controller.
\begin{align*}
\gls{ma} \tab  FCC_{est,t} &= \frac{\sum_{i=0}^{n} FCC_{pred,t-i} }{n}\\
\gls{es} \tab FCC_{est,t} &= \gls{sf} . FCC_{pred,t} + (1-sf) . FCC_{est,t-1}
\end{align*}
MA lost beide problemen goed op. Meer voorspellingen leidt tot stabielere schattingen, maar dit introduceert een vertraging (lag) op de cadans. I.e. wanneer de omstandigheden veranderen, zal de ingestelde cadans slechts na enkele iteraties optimaal zijn, in plaats van onmiddellijk. ES verminderd de amplitude van de oscillaties slechts in mindere mate. De onderstaande figuren tonen wat de impact is van ES en MA op ruizige voorspellingen (20\% kans op ruis tussen -10 en +10).
\begin{figure}[t!]
\centering
\begin{subfigure}{.5\textwidth}
  \centering
  \includegraphics[width=\linewidth]{images/actual-prediction+noice,nopp.png}
  \caption{Geen postprocessing}
  \label{fig:geen postprocessing}
\end{subfigure}%
\begin{subfigure}{.5\textwidth}
  \centering
  \includegraphics[width=\linewidth]{images/actual-prediction+noice,es.png}
  \caption{Exponential Smoothing}
  \label{fig:exponential smoothing postprocessing}
\end{subfigure}
\begin{subfigure}{.5\textwidth}
  \centering
  \includegraphics[width=\linewidth]{images/actual-prediction+noice,ma.png}
  \caption{Moving Average}
  \label{fig:moving average postprocessing}
\end{subfigure}
\caption{De effecten van verschillende postprocessing technieken.}
\label{fig:effecten postprocessing}
\end{figure}


\chapter{Resultaten}
De algoritmes worden getest in een on-line situatie gegenereerd door de simulatie. Er is enkel ruis toegevoegd op het koppel geleverd door de fietser. De modellen beginnen van nul. Gedurende een startperiode leren de algoritmes bij en zal de cadans bepaald worden door het fietsersmodel. Na de startperiode nemen de algoritmes deze instelling over. Na elke voorspelling wordt de mean squared error berekend tussen de voorspelling en het fietsersmodel. Elke 30 iteraties wordt er afgewogen of de algoritmes te ver afwijken van het fietsersmodel. Wanneer de absolute fout de grens van 5 rpm overschrijdt, zal er geleerd worden. De data gebruikt om bij te leren bestaat uit de toestand van de fiets van de afgelopen 100 iteraties (10s). Deze data wordt verwerkt tot een set van 50 training instanties, zijnde data van iteratie 0-49,1-50,..., 50-99. Deze set wordt toegevoegd aan de trainingsset die gebruikt wordt door de algoritmes. Met deze evaluatiemethode wordt er nagegaan hoe snel en hoe vaak het algoritme bijleert. Alle algoritmes worden geëvalueerd in exact dezelfde omstandigheden.
\section{Sequentie preprocessing}
De lengte van de sequenties heeft een invloed op de resultaten. Zoals te zien op figuur \ref{fig:seqlen error} heeft een te kleine sequentie negatieve invloeden op de resultaten van PA. Hoe groter de lengte van de sequentie, hoe accurater de voorspellingen worden. De tijd die het algoritme nodig heeft om de testen te voltooien stijgt ook naar gelang de grootte van de sequenties. De error en uitvoeringstijd bij DT en RF ondervinden een kleine impact bij het variëren van de lengte van de sequenties. 
\\\\
Om goede resultaten te krijgen zetten we de lengte van de sequenties boven de 20. Wat juist de beste optie is, is moeilijk te zeggen. Een kleine sequentie omvat maar enkele omwentelingen van de pedalen. Als er hier een grote inconsistentie voordoet kan dit slechte resultaten opleveren. Daarom zullen alle tests vanaf dit punt een constant lengte van 50 hebben. 
\begin{figure}[t!]
\centering
\begin{subfigure}{.49\textwidth}
  \centering
  \includegraphics[width=\linewidth]{images/evaluatie/seqlen10.png}
\end{subfigure}
\begin{subfigure}{.49\textwidth}
  \centering
  \includegraphics[width=\linewidth]{images/evaluatie/seqlen20.png}
\end{subfigure}
\begin{subfigure}{.49\textwidth}
  \centering
  \includegraphics[width=\linewidth]{images/evaluatie/seqlen50.png} 
\end{subfigure}
\begin{subfigure}{.49\textwidth}
  \centering
  \includegraphics[width=\linewidth]{images/evaluatie/seqlen100.png}   
\end{subfigure}
\caption{De invloed van sequentielengte op de error}
\label{fig:seqlen error}
\end{figure}
\section{Algoritmes}
\subsection{Passive Aggressive Algorithm}
Er zijn enkele hyperparameters die ingesteld kunnen worden. $Max\_ iter$ is het maximum aantal iteraties dat het algoritme probeert bij te leren. Tol is een parameter die bepaalt of het algoritme vroegtijdig stopt. Dit gebeurt wanneer de fout na een leercyclus met minder dan tol verbeterd. In de testomgeving, met $max\_ iter=25$ en $tol=0.1$, wordt deze vroegtijdige stop altijd behaalt. Met andere woorden, na 25 iteraties heeft het algoritme de trainingsdata geleerd.
\\\\
Een laatste interessante parameter is C: de agressiviteit parameter. Hoe hoger deze is, hoe agressiever het algoritme de gewichten gaat bijwerken. De onderstaande figuur toont hoe de C parameter de mean squared error beïnvloed van zowel PA-I als PA-II. C is de enige parameter die aangepast wordt.
\\\\
Bijna alle tests convergeren naar een mse tussen tien en vijftien, met een gemiddelde tussen dertien en veertien. Wat het verschil is tussen PA-I en PA-II valt niet duidelijk te zien. In de paper van Crammer et al. \cite{pa algorithm} worden beide versies vergeleken op basis van instance noise en label noise. In beide gevallen scoren PA-I en PA-II aanzienlijk beter dan het standaard algoritme. In dit experiment scoren PA-I en PA-II gelijkaardig.
\\\\
Figuur \ref{fig:gemiddeld aantal keer trainen pa} toont de invloed van C op het aantal keer trainen. Het verschil tussen de verschillende settings is klein. De “stappen” die genomen worden tijdens het trainen zullen dus vaak klein genoeg zijn zodat C=1 agressief genoeg is. Het is dus niet nodig om hoge C-waarden (5-10) te gebruiken. Deze data is genomen over tien sessies, van elk 20000 iteraties, en duurde telkens gemiddeld 90 seconden. 
\\\\
\begin{figure}[h]
	\includegraphics[width=\linewidth]{images/evaluatie/gemiddeldmsepa.png}
	\caption{De invloed van C op mse van PA}
	\label{fig:invloed C op PA}
\end{figure}
\newpage
\begin{figure}[t]
	\includegraphics[width=\linewidth]{images/evaluatie/aantalkeertrainenpa.png}
	\caption{Gemiddeld aantal keer trainen PA}
	\label{fig:gemiddeld aantal keer trainen pa}
\end{figure}
\subsection{Decision Tree en Random Forest}
De belangrijkste parameter bij deze rule-based learners is de $max\_ depth$. Dit is een moeilijk in te schatten parameter, vooral voor de DT. Diepere DT’s maken betere voorspellingen, maar op een bepaald punt begint de DT te overfitten. Bij RF kan ook het aantal bomen ingesteld worden. Hoe meer bomen er gebruikt worden, hoe minder invloed ruis heeft op voorspellingen en hoe beter de voorspellingen worden.
\\\\
DT’s en RF’s convergeren beide naar ongeveer dezelfde mean squared error. Diepere bomen leiden evident naar een lagere mean squared error. Algemeen zijn grotere RF’s beter, maar in deze situatie is het verschil klein.
\\\\
Voor DT, daalt het gemiddeld aantal trainingen spectaculair tussen diepte drie en vier (figuur \ref{fig:invloed diepte en aantal bomen trainen}). Een DT van diepte drie zal dus geen goede oplossing zijn. RF’s daarentegen presteren wel goed met een diepte van drie. Een DT/RF dieper dan vier is niet nodig, aangezien dit geen groot voordeel oplevert en mogelijk overfit.
\\\\
De uitvoeringstijd (figuur \ref{fig:invloed diepte en aantal bomen uitvoeringstijd}) lijkt geen probleem te vormen voor grotere RF. Het verschil tussen de uitvoeringstijden van een RF met diepte drie en de andere is te wijten aan het aantal keer dat getraind moest worden op diepte drie.

\begin{figure}[t!]
\centering
\begin{subfigure}{\textwidth}
  \centering
    \hspace*{-0.75cm}                                                           
  \includegraphics[width=\linewidth]{images/evaluatie/gemiddeldmsedt.png}
\end{subfigure}
\begin{subfigure}{\textwidth}
  \centering
  \includegraphics[width=\linewidth]{images/evaluatie/gemiddeldmserf.png}
\end{subfigure}
\caption{De invloed van diepte en aantal bomen op de gemiddelde mean squared error van DT en RF (10 keer 20 iteraties)}
\label{fig:invloed diepte en aantal bomen mse}
\end{figure}
\begin{figure}[t!]
\centering
\begin{subfigure}{\textwidth}
  \centering
    \hspace*{-0.75cm}                                                           
  \includegraphics[width=\linewidth]{images/evaluatie/aantalkeertrainendt.png}
\end{subfigure}
\begin{subfigure}{\textwidth}
  \centering
  \includegraphics[width=\linewidth]{images/evaluatie/aantalkeertrainenrf.png}
\end{subfigure}
\caption{De invloed van diepte en aantal bomen op het gemiddeld aantal keer trainen van DT en RF (10 keer 20 iteraties)}
\label{fig:invloed diepte en aantal bomen trainen}
\end{figure}
\begin{figure}[t!]
\centering
\begin{subfigure}{\textwidth}
  \centering
  \includegraphics[width=\linewidth]{images/evaluatie/uitvoeringstijddt.png}
\end{subfigure}
\begin{subfigure}{\textwidth}
  \centering
  \includegraphics[width=\linewidth]{images/evaluatie/uitvoeringstijdrf.png}
\end{subfigure}
\caption{De invloed van diepte en aantal bomen op de uitvoeringstijd van DT en RF (10 keer 20 iteraties)}
\label{fig:invloed diepte en aantal bomen uitvoeringstijd}
\end{figure}

\chapter{Discussie}
\section{Algoritmes}
Het vorige hoofdstuk toont de resultaten van de verschillende algoritmes. RF en DT presteren duidelijk beter dan beide types van PA. DT presteert goed, maar is nog steeds bekend om het slecht omgaan met ruis en het overfitten. Aangezien het verschil in uitvoeringstijd tussen DT en RF minimaal is, is het altijd beter om een RF te gebruiken. Alhoewel in de test blijkt dat de grootte van het RF geen al te grote impact heeft op de uitvoeringstijd, kan dit mogelijk nog een probleem vormen door de beperkte rekenkracht van de Raspberry Pi. Het controleprogramma voor de fiets zal immers parallel lopen met de cadanscontroller. Gelukkig heeft de Raspberry Pi vier cores en is een RF makkelijk te parallelliseren (door het instellen van het aantal parallelle taken).
\\\\
Voor een RF zijn volgende hyperparameters aangeraden:
\begin{gather*}
\text{RF \tab diepte=4 of 5, aantal bomen=10-20}
\end{gather*}
Een groot nadeel van RF is dat het niet goed om kan met ongeziene omstandigheden. Bijvoorbeeld als het algoritme van niets of weinig data begint, kan het niet uit de reeds geziene data een goede voorspelling maken (t.o.v. lineaire regressie voor een simpel lineair probleem).
\section{Waarheid}
In de verschillende testen werd het fietsersmodel altijd als waarheid (\textit{ground truth}) gezien. Dit kan echter niet gebruikt worden aangezien dit ongekend is. Als de fietser een update wilt doen, dan weet het algoritme niet hoe hard het algoritme moet aanpassen. Als de fietser sneller wilt trappen, is het dan twee rpm sneller of vijf? Dit is in deze thesis niet onderzocht, maar vormt nog een potentieel probleem in het succes van de real-time cadansaanpassing in een automatische fiets transmissie.
\\\\
Het voornaamste probleem dat zich hier voordoet, is dat twee achtereenvolgende trainingen data zullen genereren die in conflict gaan met elkaar. Stel dat de fietser 60 rpm trapt. De eerste keer dat hij op de knop duwt, wordt er trainingsdata gegenreerd met als doelvariabele 65 rpm. De tweede keer wordt er gelijkaardige data gegenereerd, maar dan met een doelvariabele van 70 rpm. Een mogelijke oplossing voor dit probleem, is het kijken naar de dichtst bijzijnde buren in de trainingsset en de doelvariabelen te updaten volgens de afstand. Hoeveel de doelvariabele bijgewerkt moet worden, zou exponentieel moeten dalen voor observaties die ver weg liggen.
\section{Conceptuele drift}
In sectie 3.4 werden de resultaten van twee algoritmes, statisch schuivend venster en bevoordeelde reservoir sampling, getoond. In beide gevallen moest er minstens drie keer zoveel geleerd worden om het nieuwe concept te leren, wat toch wel een groot verschil is met het beginnen vanaf nul. Deze technieken doen het alleszins wel beter dan geen “vergeet” algoritme. Het voornaamste probleem hier is dat mensen die deze fiets tweedehands kopen of als leen-, familiefiets gebruiken, een slechtere ervaring zullen hebben dan mensen die de fiets nieuw kopen en enkel zelf gebruiken.
\section{Verder werk}
De huidige implementatie van een RF, geïmplementeerd in de \texttt{scikit-learn} bibliotheek, doet niet aan lokale regressie in de bladeren. In plaats daarvan bevatten de bladeren een gewoon getal. Mogelijk kan lokale regressie in de bladeren een betere prestatie leveren.
\\\\
Er werd een probleem aangehaald met een RF: het kan niet goed om met ongeziene omstandigheden, zoals wanneer de fiets nog niet is ingesteld door de gebruiker. Er wordt hiervoor volgende oplossingen voorgesteld: 

\begin{itemize}
\item een standaard model voorzien
\item hetgeen wat het RF voorspelt veranderen van de eigenlijke cadans naar een verschil ten opzichte van een ingestelde basis cadans.
\end{itemize}

\noindent Een standaard model voorzien vergt extra werk. Het tweede voorstel daarentegen is makkelijk te implementeren. Trappen op een basis cadans is gelijkaardig met hoe de fiets werkt zonder cadanscontroller (mits er niet op de knoppen geduwt wordt).
\\\\
Voor het deel van conceptuele drift, zou de grootte van het venster of reservoir bepaald kunnen worden aan de hand van gebruikerstesten. Daarnaast is het uitwerken van een profielensysteem ook zeker een interessant topic om verder aan te werken. Een profielensysteem houdt verschillende modellen bij van verschillende personen. Indien iemand gaat fietsen, met een reeds bestaand profiel, zou het profielensysteem het overeenkomstige profiel moeten activeren.

\chapter{Conclusie}
Eerst werd er een realistische geparametriseerde simulatie opgezet. Die houdt rekening met verschillende lasten. De simulatie zorgt ervoor dat er gemakkelijk en snel data kan gegenereerd worden voor verschillende testen. Ten tweede werden enkele algoritmes besproken en geëvalueerd. De resultaten tonen dat PA niet geschikt is voor dit probleem, aangezien de fout te groot is en het te vaak moet bijleren. DT en RF presteerden hier goed. Ten derde werd nagegaan of de algoritmes kunnen omgaan met een stochastische update-strategie. Wederom presteert PA slecht. Binaire beslissingsbomen, zowel bij DT als RF, met diepte drie presteren hier ook slecht omdat het verschil tussen fietsersmodel en voorspelling vrij groot blijft gedurende de test. Uit deze twee testen werd er besloten dat enkel DT en RF, met diepte vier of dieper, geschikt zijn voor dit probleem. Uiteindelijk werd de keuze gemaakt om enkel verder te werken met een RF. Vervolgens werden twee technieken getest die kunnen omgaan met conceptuele drift, het veranderen van concept over tijd. De drift die hier getest werd was abrupt in deze context, omdat de \textit{freely chosen cadence} van een fietser praktisch niet veranderd over de levensduur van de fiets. De resultaten tonen aan dat kleinere structuren beter presteren dan grotere. Sampling presteert in het algemeen beter dan een statisch schuivend venster. Of omgaan met conceptuele drift noodzakelijk is, hangt af van hoe men de fiets gebruikt. Er werd ten slotte nog een probleem aangehaald met de testen en trainingsdata. In de simulatie werd een “waarheid” (\textit{ground truth}) berekend met behulp van het fietsersmodel. In realiteit bestaat dit niet en zullen updates incrementeel moeten doorgevoerd worden. Dit moet zeker nog uitgewerkt worden. Uiteindelijk moet er ook nog een implementatie voorzien worden die samenwerkt met de fietscontroller van Ellio.
\chapter*{Appendix}
\addcontentsline{toc}{chapter}{Appendix}
\section*{IntuEdrive}
\addcontentsline{toc}{section}{IntuEdrive}
Ir. Tomas Keppens werkte als departementshoofd bij Toyota toen hij in 2010 met het IntuEdrive project begon. In het kader van aan aantal masterproeven aan de KU Leuven werd het project stap per stap uitgewerkt.. In het academiejaar 2016-2017 werkte ingenieursstudent Jorrit Heidbuchel de aansturing van het intuEdrive CVT systeem uit. Halverwege 2017 was er het eerste prototype. Later dat jaar richtten Tomas Keppens en Jorrit Heidbuchel samen intuEdrive op.
\\\\
IntuEdrive wil mobiliteit duurzamer en efficiënter maken door twee-wielmobiliteit veiliger te maken. Het bedrijf bouwt en ontwikkelt E-bikes die perfect passen in het leven van zijn klanten: makkelijk in gebruik, veilig en betrouwbaar.
\\
Vandaag telt IntuEdrive 4 werknemers. CoSaR gaat in het najaar van 2019 voor het eerst in productie en mikt in 2020 op een verkoop van 1000 stuks in België.
\begin{figure}[h]
  \centering
  \includegraphics[width=0.5\linewidth]{images/logo_intuedrive.png}
  \caption{Logo IntuEdrive}
  \label{fig:Logo IntuEdrive}
\end{figure}
\newpage
\begin{thebibliography}{9}
\addcontentsline{toc}{section}{Bibliografie}
\bibitem{hardware implementation} 
Heidbuchel, J. (2017); Hardware Implementation and Control Strategy of a High Dynamic CVT Transmission for an E-Bike; KULEUVEN
 
\bibitem{randomforest paper} 
Breiman, L. (2001); Random Forests; UC BERKELEY

\bibitem{encoding cyclical features} 
London I. (2016); Encoding cyclical continuous features - 24-hour time; https://ianlondon.github.io/blog/encoding-cyclical-features-24hour-time/

\bibitem{preprocessing faq}
Sarle, W. comp.ai.neural-nets FAQ; http://www.faqs.org/faqs/ai-faq/neural-nets/part1/preamble.html

\bibitem{pa algorithm}
Crammer, Dekel, Keshet, Shalev-Shwartyz, Singer (2006); Online Passive-Aggressive Algorithms; Hebrew University of Jerusalem

\bibitem{cvt wikipedia}
https://en.wikipedia.org/wiki/Continuously\_ variable\_ transmission

\bibitem{vermogen wikipedia}
https://nl.wikipedia.org/wiki/Vermogen\_\%28natuurkunde\%29

\bibitem{random forest wikipedia}
https://en.wikipedia.org/wiki/Random\_ forest

\end{thebibliography}






\newpage
% ----------------------- Achterblad ------------------------------
% Vergeet niet de tekst aan te passen:
% - Afdeling
% - Adres van de afdeling
% - Telefoon en faxnummer
% -----------------------------------------------------------------
\thispagestyle{empty}
\sffamily
%
\begin{textblock}{191}(113,-11)
{\color{blueline}\rule{160pt}{5.5pt}}
\end{textblock}
%
\begin{textblock}{191}(168,-11)
{\color{blueline}\rule{5.5pt}{59pt}}
\end{textblock}
%
\begin{textblock}{183}(-24,-11)
\textblockcolour{}
\flushright
\fontsize{7}{7.5}\selectfont
\textbf{Computerwetenschappen}\\
Celestijnenlaan 200 A bus 2402\\
3000 LEUVEN, BELGI\"{E}\\
tel. + 32 16 32 77 00\\
fax + 32 16 32 79 96\\
www.kuleuven.be\\
\end{textblock}
%
\begin{textblock}{191}(154,-7)
\textblockcolour{}
\includegraphics*[height=16.5truemm]{sedes}
\end{textblock}
%
\begin{textblock}{191}(-20,235)
{\color{bluetitle}\rule{544pt}{55pt}}
\end{textblock}
\end{document}
